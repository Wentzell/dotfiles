\documentclass[11pt]{scrreprt}

\usepackage{fontenc}

\renewcommand\familydefault{\sfdefault}
\usepackage[slantedGreek]{sfmath}

\usepackage{graphicx}
\usepackage{epstopdf} %use for .eps in pdf
\usepackage{color,amsmath,amssymb,amsfonts,mathrsfs,bbm}
\usepackage{afterpage,fancyhdr,tabularx}
\usepackage{geometry}
\usepackage{cite}

\geometry{paper=a4paper,left=3.25cm,right=3.25cm,top=3.3cm,bottom=3.7cm,dvips}
% \geometry{paper=a4paper,left=3.6cm,right=2.9cm,top=3.3cm,bottom=3.7cm,twoside,dvips}
\setlength{\parindent}{2ex}

\unitlength 1cm

\setcounter{secnumdepth}{2}
\setcounter{totalnumber}{6}
\setcounter{topnumber}{6}
\setcounter{bottomnumber}{6}

%\setcounter{tocdepth}{0}

\interfootnotelinepenalty=10000

\usepackage{dpfloat}

\usepackage{color}
\definecolor{darkred}{rgb}{0.90,0,0}
\definecolor{darkgreen}{rgb}{0,0.60,.2}
\definecolor{darkblue}{rgb}{0,0,1}
\definecolor{grey}{cmyk}{0,0,0,0.25}
\definecolor{orange}{cmyk}{0,0.6,0.8,0}
\newcommand{\red}{\color{darkred}}
\newcommand{\blue}{\color{darkblue}}
\newcommand{\black}{\color{black}}
\newcommand{\green}{\color{darkgreen}}
\newcommand{\grey}{\color{grey}}
\newcommand{\orange}{\color{orange}}
\newcommand{\white}{\color{white}}
\definecolor{blueP}{rgb}{0.0,0.08,0.45}

\usepackage[hang]{caption}

\renewcommand*\figurename{Figure}
\renewcommand*\tablename{Table} %%%%%%% CHANGED BY NILS
\renewcommand{\bottomfraction}{.8}
\renewcommand{\topfraction}{.8}

\DeclareCaptionFont{blueP}{\color{blueP}}
\captionsetup{font=small,labelfont={bf,blueP}}

\pagestyle{fancy}
\fancyhead[LE]{\color{blueP}\sffamily\thepage\hspace{1cm}\leftmark}
\fancyhead[RO]{\color{blueP}\sffamily\rightmark\hspace{1cm}\thepage}
\fancyhead[LO,RE]{}
\fancyfoot{}

\renewcommand{\sectionmark}[1]{\markright{\textsl{\thesection.\ {#1}}}}

\usepackage{titlesec}
% \titleformat{⟨Überschriftenklasse⟩}[Absatzformatierung⟩]{⟨Textformatierung⟩} {⟨Nummerierung⟩}{⟨Abstand zwischen Nummerierung und Überschriftentext⟩}{⟨Code vor der Überschrift⟩}[⟨Code nach der Überschrift⟩]
% \titlespacing{⟨Überschriftenklasse⟩}{⟨Linker Einzug⟩}{⟨Platz oberhalb⟩}{⟨Platz unterhalb⟩}[⟨rechter Einzug⟩]
\titleformat{\chapter}[display]{\normalfont\huge\bfseries\color{blueP}}{\chaptertitlename\ \thechapter}{20pt}{\huge}
\titleformat{\section}{\normalfont\Large\bfseries\color{blueP}}{\thesection}{0.8em}{}
\titleformat{\subsection}{\normalfont\large\bfseries\color{blueP}}{\thesubsection}{0.8em}{}
\titleformat{\subsubsection}{\normalfont\normalsize\bfseries\color{blueP}}{\thesubsubsection}{1em}{}
\titlespacing*{\chapter}{0pt}{50pt}{20pt}
\titlespacing*{\section}{0pt}{2.75ex plus 0.7ex minus .2ex}{1.75ex plus .2ex}
\titlespacing*{\subsection}{0pt}{2.5ex plus 0.7ex minus .2ex}{1.2ex plus .2ex}
\titlespacing*{\subsubsection}{0pt}{2ex plus 0.7ex minus .2ex}{1ex plus .2ex}

\numberwithin{equation}{section}

\usepackage[undotted]{minitoc}
\mtcsetfont{minitoc}{subsection}{\normalsize\sffamily}
\mtcsetfont{minitoc}{section}{\normalsize\bfseries\sffamily}
\renewcommand{\mtifont}{\large\sffamily\bfseries\color{blueP}}

\newcommand{\ctstyle}[1]{{\small #1}}
\newcommand{\ctstyleFN}[1]{{\footnotesize #1}}
\renewcommand\citeleft{{\color{blueP}[}}
\renewcommand\citeright{{\color{blueP}]}}
\renewcommand\citemid{{\color{blueP},}}
\renewcommand\citeform[1]{\color{blueP} #1}

\newcommand{\mc}{\mathcal}
\newcommand{\la}{\Lambda}
\newcommand{\tn}{\textnormal}

\newcommand{\cprb}[4]{\textsl{`#1'}, Phys.~Rev.~B {\bf #2}, #3 (#4)}
\newcommand{\cprl}[4]{\textsl{`#1'}, Phys.~Rev.~Lett.~{\bf #2}, #3 (#4)}
\newcommand{\cnjp}[4]{\textsl{`#1'}, New J.~Phys.~{\bf #2}, #3 (#4)}
\newcommand{\crevmod}[4]{\textsl{`#1'}, Rev.~Mod.~Phys.~{\bf #2}, #3 (#4)}
\newcommand{\cjp}[4]{\textsl{`#1'}, J.~Phys.: Condensed Matter {\bf #2}, #3 (#4)}
\newcommand{\cjpa}[4]{\textsl{`#1'}, J.~Phys.~A {\bf #2}, #3 (#4)}
\newcommand{\cnature}[4]{\textsl{`#1'}, Nature {\bf #2}, #3 (#4)}
\newcommand{\cbook}[2]{\textsl{`#1'} (#2)}

\newcommand{\refeq}[1]{Eq.~(\ref{#1})}
\newcommand{\refsec}[1]{Section~\ref{#1}}
\newcommand{\reffig}[1]{Figure~\ref{#1}}
\newcommand{\refcha}[1]{Chapter~\ref{#1}}
\newcommand{\refapp}[1]{Appendix~\ref{#1}}
\newcommand{\reftab}[1]{Table~\ref{#1}}
\newcommand{\fa}{{\small(}a{\small)}}
\newcommand{\faa}{{\footnotesize(}a{\footnotesize)}}
\newcommand{\fb}{{\small(}b{\small)}}
\newcommand{\fbb}{{\footnotesize(}b{\footnotesize)}}
\newcommand{\fc}{{\small(}c{\small)}}

\newcommand{\monthword}[1]{\ifcase#1\or Januar\or Februar\or M\"arz\or April\or
                                        Mai\or Juni\or Juli\or August\or
                                        September\or Oktober\or November\or Dezember\fi}

\newcommand{\PDFpagebreak}{\pagebreak}

\makeatletter
\newcommand{\fmslash}[2][0mu]{%
 \mathchoice
   {\fmsl@sh\displaystyle{#1}{#2}}%
   {\fmsl@sh\textstyle{#1}{#2}}%
   {\fmsl@sh\scriptstyle{#1}{#2}}%
   {\fmsl@sh\scriptscriptstyle{#1}{#2}}}
\newcommand{\fmsl@sh}[3]{%
 \m@th\ooalign{$\hfil#1\mkern#2/\hfil$\crcr$#1#3$}}
\makeatother

\usepackage[  pdfdisplaydoctitle=true, pdfborder={0 0 0},
              pdfpagelayout=SinglePage, pdfpagemode=UseOutlines, pdfstartview=Fit, %implicit=false,
pdftitle={Magnetic field effects on the Andreev bound states in Josephson quantum dots},
pdfauthor={Nils Wentzell},
pdfcreator={pdfTeX using libpoppler 3.1415926-1.40.9-2.2 (Web2C 7.5.7)}   ]{hyperref}

%%%%%%%%%%%%%%%%%%%%%%%%%%%%%%%%%%%%%%%%%%
%%%%%%		CHRISTOPH ADDED HERE	%%%%%%
%%%%%%%%%%%%%%%%%%%%%%%%%%%%%%%%%%%%%%%%%%
\newcommand{\diagramheight}{65pt}


\newcommand{\ket}[1]{\left|#1\right\rangle}
\newcommand{\bra}[1]{\left\langle#1\right|}
\newcommand{\braket}[2]{\left\langle#1\right.\left|#2\right\rangle}
\renewcommand{\vec}[1]{\underline{#1}}
\newcommand{\commutator}[2]{\left[#1,#2\right]}
\newcommand{\average}[1]{\langle#1\rangle}
\newcommand{\tr}{\text{Tr}}
\newcommand{\im}[1]{\text{Im }#1}
\newcommand{\re}[1]{\text{Re }#1}
\newcommand{\logTerm}[2]{\mathcal L_{#1}(#2)}
\newcommand{\abs}[1]{{\left\lvert{#1}\right\rvert}}
\newcommand{\ii}{\mathrm{i}}
\newcommand{\jsquare}[1]{\left(J^{#1}_{\alpha\beta}\right)^2}
\newcommand{\diag}[1]{diag\left\lbrace #1 \right\rbrace}
\renewcommand{\hat}[1]{\widehat{#1\,\,}}
\newcommand{\refschoeller}[1]{Eq.~(#1) from~\refcite{Schoeller2009}}
\newcommand{\deltaV}{\alpha\delta_{\alpha\bar\beta}V}
\newcommand{\refcite}[1]{Ref.~\cite{#1}}

%\newcommand{\makephantomsection}[1]{ % Wahl ob Phantomsections im Dokument oder nur im toc auftauchen
%	\phantomsection
%	\part*{\color{blueP}#1\addcontentsline{toc}{paragraph}{#1}}
%}
 \newcommand{\makephantomsection}[1]{
 	\newpage
 	\addcontentsline{toc}{paragraph}{#1}
 }

%\renewcommand{\arraystretch}{1.1}
\usepackage[nottoc]{tocbibind}
\settocbibname{Bibliography}

\usepackage[numbers]{natbib} 	
\bibliographystyle{apsrev4-1}

\usepackage{booktabs}

\usepackage{amsmath}
\usepackage{amssymb}
\usepackage{amsfonts}
\usepackage{dsfont}
\usepackage{simplewick} % http://www.fzu.cz/~kolorenc/tex/simplewick/simplewick.pdf
\usepackage{appendix}  % http://www.ctan.org/tex-archive/help/Catalogue/entries/appendix.html

%%%%%%%%%%%%%%%%%%%%%%%%%%%%%%%%%%%%%%%%%%%
% Nils added here
%%%%%%%%%%%%%%%%%%%%%%%%%%%%%

\renewcommand{\Re}{\operatorname{Re}}
\renewcommand{\Im}{\operatorname{Im}}
\newcommand{\Tr}{\operatorname{Tr}}

\usepackage{grffile}
